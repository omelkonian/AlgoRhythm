\documentclass[12pt,a4paper]{article}
 
% Timeline
\usepackage{chronosys}

% Code
\usepackage{minted}

% Tables
\usepackage{multicol}
\usepackage{multirow}
\usepackage{csquotes}
\usepackage{fullpage}

% Colors
\usepackage{xcolor, color, colortbl}
\colorlet{gray}{gray!70}
\colorlet{green}{green!50}
\definecolor{darkblue}{HTML}{1D577A}
\definecolor{rred}{HTML}{C03425}
\definecolor{darkgreen}{HTML}{8BB523}
\definecolor{ppurple}{HTML}{6B1B7F}
\definecolor{pblack}{HTML}{000000}
\definecolor{darkyellow}{HTML}{C0B225}

% Links
\usepackage{hyperref}
\definecolor{linkcolour}{rgb}{0,0.2,0.6}
\hypersetup{colorlinks,breaklinks,urlcolor=linkcolour,linkcolor=linkcolour,citecolor=blue}

% Title
\title{\textbf{Project Proposal \\ \small{Advanced Functional Programming}}}
\author{\small{Joris ten Tusscher, Cas van der Rest, Orestis Melkonian}}
\date{}

% Macros
\newcommand{\site}[1]{\footnote{\url{#1}}}
\newcommand{\code}[1]{\mintinline{bash}{#1}}

\begin{document}
\maketitle

\section{Domain}

\subsection{Algorithmic Music Composition}
\subsection{Generation Techniques}
\subsection{Motivation}
\cite{categorial}

\section{Problem}

\subsection{Music-Representation DSL}
\subsubsection{Euterpea}
Euterpea\site{https://hackage.haskell.org/package/Euterpea}
\subsubsection{Export to MIDI}
Midi\site{http://hackage.haskell.org/package/midi}
\subsubsection{Render to music scores}
Lilypond\site{https://hackage.haskell.org/package/lilypond}

\subsection{Generation DSL}
\subsubsection{Chaos Functions}
\cite{chaos}
\subsubsection{L-Systems}
\cite{lsystem}
\subsubsection{QuickCheck}

\subsection{Constraint DSL}
As the solution space defined by our categorial grammar alone is huge, searching for solutions exhibiting specific desired properties (e.g. melodies involving notes from a certain scale) would be computationally infeasible.

To remedy this, we will implement a DSL that will allow the programmer to naturally express constraints, which will be respected by the musical artefacts we generate; these will model musical properties such as restricted pitch range. As you would expect, these constraints will not be applied posthumously as a filter, but integrated in the generation process, effectively pruning the search space.

\subsection{Applications}
Apart from the above, we also aim to implement several applications, showcasing the features of our library:
\vspace{-15pt}
\paragraph{Music Representation} We will provide code snippets that demonstrate one's ability to write concrete music pieces using our DSL and to export them in MIDI format or music notation.
\vspace{-15pt}
\paragraph{Generation} We plan to implement several common generation techniques, such as creating melodies from 
chaotic/complex functions and structuring pieces via an L-system grammar.
\vspace{-15pt}
\paragraph{Constraints} We will demonstrate how our library can be used for automatic generation of musical exercises, utilizing a variety of constraints.
\vspace{5pt}

An important property of our library that we wish to show through our examples, is that it is not geared specifically towards single-voice melodies, but can be used as easily to generate rhythm, harmony or anything combining these three principal elements of music.
If time permits, we will also implement a simple web interface, which runs our library on the back-end and allows the user to select a number of pre-defined constraints in order to generate, for instance, musical exercises.
Last but not least, the library will ship with its own "Prelude", providing common patterns/techniques for algorithmic music composition.

\section{Planning}
Below we give the estimated schedule across the six weeks available:
\vspace{.5cm}
\setupchronology{startyear=0, stopyear=6, startdate=false, stopdate=false, width=.9\hsize, height=.5cm, arrow=false}
\setupchronoperiode{textstyle=\bf\footnotesize}
\setupchronoevent{date=false, textstyle=\it\large, markdepth=1.5cm}
\chronoperiodecoloralternation{rred, darkgreen, rred, darkgreen}

\startchronology
\chronoevent{0}{Proposal(20/2)}
\chronoperiode{0}{2}{Music Representation}
\chronoperiode{2}{3}{Generation}
\chronoevent{3}{Report(12/3)}
\chronoperiode{3}{5}{Constraints}
\chronoperiode{5}{6}{Applications}
\chronoevent{6}{Submission(5/4)}
\stopchronology
\vspace{.5cm}

\bibliographystyle{ieeetr}
\bibliography{sources}

\end{document}

