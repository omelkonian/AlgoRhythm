\documentclass[12pt,a4paper]{article}
 
% Timeline
\usepackage{chronosys}

% Code
\usepackage{minted}

% Tables
\usepackage{multicol}
\usepackage{multirow}
\usepackage{csquotes}
\usepackage{fullpage}

% Colors
\usepackage{xcolor, color, colortbl}
\colorlet{gray}{gray!70}
\colorlet{green}{green!50}
\definecolor{darkblue}{HTML}{1D577A}
\definecolor{rred}{HTML}{C03425}
\definecolor{darkgreen}{HTML}{8BB523}
\definecolor{ppurple}{HTML}{6B1B7F}
\definecolor{pblack}{HTML}{000000}
\definecolor{darkyellow}{HTML}{C0B225}

% Links
\usepackage{hyperref}
\definecolor{linkcolour}{rgb}{0,0.2,0.6}
\hypersetup{colorlinks,breaklinks,urlcolor=linkcolour,linkcolor=linkcolour,citecolor=blue}

% Title
\title{\textbf{Project Proposal \\ \small{Advanced Functional Programming}}}
\author{\small{Joris ten Tusscher, Cas van der Rest, Orestis Melkonian}}
\date{}

% Macros
\newcommand{\site}[1]{\footnote{\url{#1}}}
\newcommand{\code}[1]{\mintinline{bash}{#1}}

\begin{document}
\maketitle

\section{Domain}
This project concerns the development of a library that aids programmers in the process of composing musical pieces. We will aim to do so by providing tools for a programmer to easily employ various techniques of algorithmic music composition through the usage of various DSL's. 
\subsection{Algorithmic Music Composition}
The notion of algorithmic music composition usually includes all methods in which music is generated through some set of predefined rules. When considering that a piece of music (in a rudimentary sense) consists of various aspects, such as harmony and rhythm, for which a clearly defined set of choices exist within their respective domain, it is easy to see that computers may generate musical pieces by employing an algorithm that decides which choices to make within these domains (e.g. which notes to play, when to play them and for how long).

\subsection{Generation Techniques}
Various approaches towards algorithmic music composition have been studied. Notable generation techniques include: 
\begin{itemize}
\item 
Chaos functions: generating notes from functions whose output is extremely sensitive to their input
\item
Generative grammars (L-systems)
\item
Evolutionary algorithms
\end{itemize}
\subsection{Motivation}
Motivation for this project is an interest in music theory as well as the aforementioned techniques for algorithmic composition of the undertakers. 

\section{Problem}
The goal of the research is to create a Haskell package that can be used to formally describe music, generate music that satisfies certain constraints specified by the user, and export music to more universal formats.

\subsection{Music-Representation DSL}
The music-representation DSL will be an EDSL in Haskell that can be used to formally describe music. It can be used to store information such as the notes present in the music piece, the musical dynamic throughout the piece (e.g. pianissimo or forte), or the key.

Since most users want to do more practical things with the music that is stored using the music-representation DSL, data written in the DSL can be converted to the Euterpea DSL\site{https://hackage.haskell.org/package/Euterpea}. Euterpea is a Haskell package developed by, among others, Paul Hudak, that can be used for music representation, but also has many other features than the music-representation DSL that will be developed for this project won't have. For example, it can analyse music and thus derive properties, perform audio synthesis, and read and export MIDI (Musical Instrument Digital Interface) data. Therefore, Euterpea would be a great tool for actually generating audio from the music that is described in our music-representation DSL, and it can also be used to export the information described in our EDSL to MIDI files. On top of Euterpea, we could use the Lilypond package\site{https://hackage.haskell.org/package/lilypond} to export the music to nice traditional music scores.

%\subsubsection{Euterpea}
%Euterpea
%\subsubsection{Export to MIDI}
%Midi\site{http://hackage.haskell.org/package/midi}
%\subsubsection{Render to music scores}
%Lilypond\site{https://hackage.haskell.org/package/lilypond}

\subsection{Generation DSL}
The generation DSL does not store actual music. Rather, it can be used for music composition. The generation can be done in multiple ways.

\subsubsection{Chaos Functions}
Chaotic functions generate completely different output as soon as their input changes by a minuscule amount. Therefore, they can be used to generate completely different music every time their input is changed by a fraction.
%\cite{chaos}
\subsubsection{Lindenmayer Systems}
L-systems \cite{lsystem-original} are a type of formal grammar invented at Utrecht University in 1968. They have an initial string and a set of production rules to expand this initial string into a longer string of symbols. The final string of symbols can be converted to something else, oftentimes trees. One can also convert the symbols to music however \cite{lsystem}.
\subsubsection{QuickCheck}
Another way to generate music is to use QuickCheck. Using custom generator and arbitrary instances, it can be made possible to generate music that satisfies the information stored using the constraint DSL.

\subsection{Constraint DSL}
As the solution space defined by our categorial grammar alone is huge, searching for solutions exhibiting specific desired properties (e.g. melodies involving notes from a certain scale) would be computationally infeasible.

To remedy this, we will implement a DSL that will allow the programmer to naturally express constraints, which will be respected by the musical artefacts we generate; these will model musical properties such as restricted pitch range. As you would expect, these constraints will not be applied posthumously as a filter, but integrated in the generation process, effectively pruning the search space.

\subsection{Applications}
Apart from the above, we also aim to implement several applications, showcasing the features of our library:
\vspace{-15pt}
\paragraph{Music Representation} We will provide code snippets that demonstrate one's ability to write concrete music pieces using our DSL and to export them in MIDI format or music notation.
\vspace{-15pt}
\paragraph{Generation} We plan to implement several common generation techniques, such as creating melodies from 
chaotic/complex functions and structuring pieces via an L-system grammar.
\vspace{-15pt}
\paragraph{Constraints} We will demonstrate how our library can be used for automatic generation of musical exercises, utilizing a variety of constraints.
\vspace{5pt}

An important property of our library that we wish to show through our examples, is that it is not geared specifically towards single-voice melodies, but can be used as easily to generate rhythm, harmony or anything combining these three principal elements of music.
If time permits, we will also implement a simple web interface, which runs our library on the back-end and allows the user to select a number of pre-defined constraints in order to generate, for instance, musical exercises.
Last but not least, the library will ship with its own "Prelude", providing common patterns/techniques for algorithmic music composition.

\section{Planning}
Below we give the estimated schedule across the six weeks available:
\vspace{.5cm}
\setupchronology{startyear=0, stopyear=6, startdate=false, stopdate=false, width=.9\hsize, height=.5cm, arrow=false}
\setupchronoperiode{textstyle=\bf\footnotesize}
\setupchronoevent{date=false, textstyle=\it\large, markdepth=1.5cm}
\chronoperiodecoloralternation{rred, darkgreen, rred, darkgreen}

\startchronology
\chronoevent{0}{Proposal(20/2)}
\chronoperiode{0}{2}{Music Representation}
\chronoperiode{2}{3}{Generation}
\chronoevent{3}{Report(12/3)}
\chronoperiode{3}{5}{Constraints}
\chronoperiode{5}{6}{Applications}
\chronoevent{6}{Submission(5/4)}
\stopchronology
\vspace{.5cm}

\bibliographystyle{ieeetr}
\bibliography{sources}

\end{document}

