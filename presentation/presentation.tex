\documentclass{beamer}
\usetheme[
  block=fill,
  background=dark,
  titleformat=smallcaps,
  progressbar=frametitle,
  numbering=none,
]{metropolis}

% Math
\usepackage{amsmath}
\usepackage{amssymb}
\usepackage{stmaryrd}

% Code listing
\usepackage{minted}
\usemintedstyle{monokai}
%\usemintedstyle{native}
%\usemintedstyle{tango}
\newcommand{\icode}[1]{\mintinline{haskell}{#1}}

% Graphics
\usepackage{graphics}
\usepackage{pdfpages}
\graphicspath{{figures/}} % Location of the graphics files

\newcommand\todo[1]{\textcolor{red}{#1}}

% Box macro
\newcommand{\ex}[2]{
  \vfill
  \begin{alertblock}{#1}
    #2
  \end{alertblock}
}
%----------------------------------------------------------------------------

% Beamer
\title{AlgoRhythm}
\subtitle{A Library for Algorithmic Music Composition}
\author{Joris ten Tusscher, Cas van der Rest, Orestis Melkonian}
\date{April 5, 2018}
\institute{Universiteit Utrecht}

\begin{document}
	\maketitle

  \begin{frame}{Some definitions}
    	\begin{itemize}
    	\item
        \textbf{Melody}: Notes played in \emph{sequence}
        \item
        \textbf{Chords/harmony}: Notes played \emph{simultaneously}
        \item
        \textbf{Scale}: a sequence of ascending notes, beginning and starting on the same note.
    	\end{itemize}

       	i.e: C major = C, D, E, F, G, A, B, C

        Or in intervals: 2, 2, 1, 2, 2, 2, 1
    \end{frame}

    \begin{frame}{Some definitions}
    	A piece of music is said to be in a \textbf{key} if it (primarily) uses notes from a certain scale

        \textbf{Diatonic} music is music that uses scales that have the same pattern as we saw before (2,2,1,2,2,2,1).
    \end{frame}

    \begin{frame}[fragile=singleslide]{Music DSL: Representation}
	Basically, you want to know when to make noise and when to remain silent.

    Two pieces of music can be composed in parallel or sequentially.
    \begin{minted}[baselinestretch=1, fontsize=\small, autogobble]{haskell}
type Duration = Rational

data Music a = Music a :+: Music a
             | Music a :=: Music a
             | Note Duration a
             | Rest Duration
    \end{minted}
	\end{frame}

    \begin{frame}[fragile=singleslide]{Music DSL: Representation}
    In order to provide export functionalities, we use a \icode{MusicCore} type and a typeclass \icode{ToMusicCore}.
    \begin{minted}[baselinestretch=1, fontsize=\small, autogobble]{haskell}
type PitchClass = C | Cs | D ... As | B
type Octave = Oct0 | Oct1 ... Oct5 | Oct6
type PitchAttribute = Dynamic Dynamic
                    | Articulation Articulation

type MusicCore =
  Music ((PitchClass, Octave), [PitchAttribute])
    \end{minted}

    This ensures that all the necessary information is there when exporting a piece of music
	\end{frame}
    \begin{frame}[fragile=singleslide]{Music DSL: Representation}
    (Abstract) scales and chords are represented as intervals between notes, i.e:

    \icode{major = [P1,M2,M3,P4,P5,M6,M7] -- Major scale}
    \icode{d7b5 = [P1, M3, A4, Mi7] -- Half diminished chord}

    There are many constants for various scales and chords (, as well as common durations:

    \begin{minted}[baselinestretch=1, fontsize=\small, autogobble]{haskell}
qn = 1%4
    \end{minted}
	\end{frame}

    \begin{frame}[fragile=singleslide]{Music DSL: Manipulation}
	Music can be constructed and manipulated using various operators

    \begin{minted}[baselinestretch=1, fontsize=\small, autogobble]{haskell}
 -- quarter note C in the 4th octave, played softly
 let n = (C#4 <: [PPP]) <| qn

 -- A half note rest
 let r = (hn~~)

 -- Instantiate an abstract chord
 let cMaj7 = ((C =| maj7) <#) 3 <|| wn
    \end{minted}
	\end{frame}

    \begin{frame}[fragile=singleslide]{Music DSL: Manipulation}
	A melody in our DSL:

    \begin{minted}[baselinestretch=1, fontsize=\small, autogobble]{haskell}
 line [ (C#6 <: [Dynamic PP]) <| wn
      , (D#6 <: [Dynamic MP]) <| wn
      , (hn~~)
      , (C#6 <: [Dynamic F_]) <| qn
      , (D#6 <: [Dynamic F_]) <| qn
      , (C#6 <: [Dynamic F_]) <| qn
      , (B#5 <: [Dynamic F_]) <| qn
      , (D#6 <: [Dynamic MF]) <| qn
      , (C#6 <: [Dynamic MP]) <| wn
      , (F#5 <: [Dynamic P ]) <| wn
      ]
    \end{minted}
	\end{frame}

    \begin{frame}[fragile=singleslide]{Music DSL: Manipulation}
	There's also some operators for common operations:

    \begin{minted}[baselinestretch=1, fontsize=\small, autogobble]{haskell}
 -- Transposition
 C ~> M3 == E

 -- Retrograded (mirroring)
 let music' = (music><)

 -- Time scaling
 let music' = music *~ (1%5)

    \end{minted}

	Also, \icode{Music} is a functor!

	\begin{minted}[baselinestretch=1, fontsize=\small, autogobble]{haskell}
	let rhythm = const () <$> music
    \end{minted}
	\end{frame}

	{\usebackgroundtemplate{%
  	\includegraphics[width=\paperwidth,height=\paperheight]{no-analysis.png}}
	\begin{frame}{Focus on Generation, Ignore Analysis}
	\end{frame}
	}

	\begin{frame}{Generation}
	\todo{genState, selectors, diatonic improv, etc...}
	\end{frame}

	\begin{frame}{Dynamic Performance}
	\todo{k-means, etc...}
	\end{frame}

	\begin{frame}[fragile=singleslide]{Grammars: Properties}
    (Generative) \textit{context-free grammars}, with a few extra features:
	\begin{itemize}
	\item \textbf{Temporal}: Rules are parametric to duration
	\item \textbf{Probabilistic}: Rules can be assigned weights
	\item \textbf{Graph}: Allow node sharing (using \textit{let}-expressions)
	\end{itemize}
	\end{frame}

	\begin{frame}[fragile=singleslide]{Grammars: Definition}
	\begin{minted}[baselinestretch=1, fontsize=\small, autogobble]{haskell}
data Grammar meta a =
    a |: [Rule meta a]
data Rule meta a =
    (a, Weight, Dur -> Bool) :-> (Dur -> Term meta a)
data Term meta a =
    a %: Dur
    | Term meta a :-: Term meta a
    | Aux Bool meta (Term meta a)
    | Let (Term meta a) (λb. Term () b -> Term () b)

class Expand meta a b | meta a -> b where
    expand :: Term meta a -> IO (Term () b)

(a, w) -| f = (a, w, f) :-> (a %:)
a |->  b = a :-> const b
a |--> b = (a, 1, always) |-> b
($:)  = Aux False
(|$:) = Aux True
	\end{minted}
	\end{frame}

	\begin{frame}[fragile=singleslide]{Grammars: Tabla Rhythm}
	\begin{minted}[baselinestretch=1, fontsize=\small, autogobble]{haskell}
tabla :: Grammar () Syllable
tabla = S |:
  [ S  |--> TE1 :-: XI
  , XI |--> TA7 :-: XD
  , XD |--> TA8
  , XG |--> TB2 :-: XA
    ...
  , TE4 |--> Ti :-: Rest :-: Dha :-: Ti
  , TC2 |--> Tira :-: Kita
  , TB3 |--> Dha :-: Tira :-: Kita
  , TD1 |--> Rest
    ...
  ]
instance ToMusicCore Syllable where
    ...
	\end{minted}
	\end{frame}

	\begin{frame}[fragile=singleslide]{Grammars: Tonal Harmony}
	\begin{minted}[baselinestretch=0.8, fontsize=\footnotesize, autogobble]{haskell}
harmony :: Grammar Modulation Degree
harmony = I |:
  [ -- Turn-arounds
    (I,  8, (> wn)) :-> \t ->
      Let (I%:t/2) (\x -> x :-: x)
  , (I,  6, (> hn) /\ (<= wn)) :-> \t ->
      II%:t/4 :-: V%:t/4 :-: I%:t/2
  , (I,  2, (> hn) /\ (<= wn)) :-> \t ->
      V%:t/2 :-: I%:t/2
  , (I,  2) -| (<= wn)
    -- Modulations
  , (V,  5, (> hn)) :-> \t -> Modulation P5 $: I%:t
  , (V,  3) -| always
  , (II, 2, (> hn)) :-> \t -> Modulation M2 |$: I%:t
  , (II, 8) -| always
  ]

instance Expand Degree Modulation SemiChord where
    ...

voiceLead :: Music SemiChord -> IO (Music Chord)
	\end{minted}
	\end{frame}

	\begin{frame}[fragile=singleslide]{Grammars: Jazz Improvisation}
	\begin{minted}[baselinestretch=0.8, fontsize=\footnotesize, autogobble]{haskell}
melody :: Grammar () NT
melody = MQ |:
  [ -- Abstract Rhythm { MQ ~> Q }
    (MQ,  1, (== qn)) |-> Q%:qn
  , (MQ, 25, (> (hn^.))) :-> \t -> Q%:hn :-: MQ%:(t - hn)
    ...
    -- Concrete Rhythm { Q ~> MN }
  , (Q, 47, (== wn)) |-> MN%:qn :-: Q%:hn :-: MN%:qn
  , (Q,  6, (== hn)) |->
      MN%:(qn^^^) :-: MN%:(qn^^^) :-: MN%:(qn^^^)
    ...
    -- Abstract Melody { MN ~> N }
  , (MN, 1, (== wn)) |-> N%:qn :-: N%:qn :-: MN%:hn
  , (MN, 1, (== qn)) |->
      N%:(en^^^) :-: N%:(en^^^) :-: N%:(en^^^)
    ...
    -- Concrete Melody { N ~> NT }
  , (N, 50, (== qn)) |-> ColorTone%:qn
  , (N, 45, (== qn)) |-> Rest%:qn
  , (N,  1, (== en)) |-> ApproachTone%:en
    ...
  ]

mkSolo :: Music SemiChord -> Music NT -> IO Melody
    \end{minted}
	\end{frame}

	\begin{frame}[fragile=singleslide]{Demo: Code}
	\begin{minted}[baselinestretch=0.9, fontsize=\footnotesize, autogobble]{haskell}
    orientalAlgebras = do
      let ?config = MusicConfig
        { basePc  = A
        , baseOct = Oct3
        , baseScale = arabian
        , chords  = equally allChords
        , scales  = equally allScales
        , octaves = [(20, Oct4), (15, Oct5), (5, Oct6)]
        , colorWeight = 0, ...
        , tempo = 6%5
        , instruments = [Piano, Sitar, Tabla]
        , beat = sn
        }
      let t = 12 * wn
      har <- voiceLead  <$> runGrammar harmony t
      mel <- mkSolo har <$> runGrammar melody  t
      rhy <- runGrammar tabla t
      writeToMidiFile "out.mid" (dyn (har :=: mel :=: rhy))
	\end{minted}
  	\end{frame}

	\begin{frame}[c]{Music score}
	\centering
	\vspace{-1.2cm}
	\begin{center}
      \makebox[\textwidth]{\includegraphics[width=.73\paperwidth,angle=-90]{oriental.pdf}}
    \end{center}
	\end{frame}

\end{document}
